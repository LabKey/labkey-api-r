\documentclass{article}
\usepackage[ae,hyper]{Rd}
\begin{document}
\HeaderA{Rlabkey-package}{Import data from a labkey data base into an R data frame}{Rlabkey.Rdash.package}
\aliasA{Rlabkey}{Rlabkey-package}{Rlabkey}
\keyword{package}{Rlabkey-package}
\begin{Description}\relax
This package allows the import of data from a labkey data base
into an R data frame through the use of Sql
commands or by specifying the query schema information.

Data in a labkey data base can be modified from an R session
by using the \code{insert}, \code{update}, and \code{delete} functions.
The user must have the appropriate authorization on the labkey
server in order to modify data in the data base through the use of
these functions.
\end{Description}
\begin{Details}\relax
\Tabular{ll}{
Package: & Rlabkey\\
Type: & Package\\
Version: & 0.0.1\\
Date: & 2008-08-18\\
License: & Apache 2.0\\
LazyLoad: & yes\\
}
Using this package to access a password protected labkey data base requires that the user
has their login information in a.netrc file. The .netrc file
contains configuration and autologin information for the File Transfer Protocol client (ftp).
The file should be located in your home directory and the permissions on the file should be unreadable for 
everybody except the owner. Permissions can be set with the chmod command from the unix command line
as chmod 600 .netrc.  ***Insert how to do this for windows.
The following three lines must be in your .netrc file:
machine machinename
login mylogin
password mypassword

An example would be:
machine atlas.scharp.org
login vobencha@fhcrc.org
password mypassword

See http://linux.about.com/library/cmd/blcmdl5_netrc.htm for more information on how to configure
the .netrc file.
\end{Details}
\begin{Author}\relax
Valerie Obenchain
\end{Author}
\begin{References}\relax
http://www.omegahat.org/RCurl/,
http://dssm.unipa.it/CRAN/web/packages/rjson/rjson.pdf,
https://www.labkey.org/project/home/begin.view
\end{References}
\begin{SeeAlso}\relax
\code{\LinkA{labkey.selectRows}{labkey.selectRows}}, \code{\LinkA{labkey.executeSql}{labkey.executeSql}}, \code{\LinkA{makeFilter}{makeFilter}}
\end{SeeAlso}

\HeaderA{labkey.deleteRows}{Delete rows of data in a labkey database}{labkey.deleteRows}
\keyword{IO}{labkey.deleteRows}
\begin{Description}\relax
From an R session, specify which row(s) of data should be delted from the database.
\end{Description}
\begin{Usage}
\begin{verbatim}
labkey.deleteRows(baseUrl, folderPath, schemaName, queryName, toDelete, stripAllHidden = TRUE)
\end{verbatim}
\end{Usage}
\begin{Arguments}
\begin{ldescription}
\item[\code{baseUrl}] a string specifying the \code{baseUrl}for the HTTP request
\item[\code{folderPath}] a string specifying the \code{folderPath} for the HTTP request
\item[\code{schemaName}] a string specifying the  \code{schemaName} for the HTTP request
\item[\code{queryName}] a string specifying the  \code{queryName} for the HTTP request
\item[\code{toDelete}] a list containing the column name of the "key" or "data identifier" and the 
corresponding identification number of the row(s) of data to be deleted
\item[\code{stripAllHidden}] [optional] a logical value indicating whether or not to save data columns that would 
normally be hidden from user veiw. If no value is specified, no hidden columns are returned.
\end{ldescription}
\end{Arguments}
\begin{Details}\relax
A single row or multiple rows of data can be deleted at a time.
\end{Details}
\begin{Value}
Information returned to the user will include the \code{schemaName} and the \code{queryName} used in the update
as well as the number of rows affected (ie, deleted).
\end{Value}
\begin{Author}\relax
Valerie Obenchain
\end{Author}
\begin{References}\relax
http://www.omegahat.org/RCurl/, 
http://dssm.unipa.it/CRAN/web/packages/rjson/rjson.pdf,
https://www.labkey.org/project/home/begin.view
\end{References}
\begin{SeeAlso}\relax
\code{\LinkA{labkey.selectRows}{labkey.selectRows}}
\end{SeeAlso}

\HeaderA{labkey.executeSql}{Retrieve data from a labkey database using Sql commands}{labkey.executeSql}
\keyword{IO}{labkey.executeSql}
\begin{Description}\relax
Use Sql commands to specify data to be imported into R. Prior to import, data can
be manipulated through all standard Sql commands.
\end{Description}
\begin{Usage}
\begin{verbatim}
labkey.executeSql(baseUrl, folderPath, schemaName, sql, maxRows = NULL, 
                                  rowOffset = NULL, stripAllHidden = TRUE)
\end{verbatim}
\end{Usage}
\begin{Arguments}
\begin{ldescription}
\item[\code{baseUrl}] a string specifying the \code{baseUrl}for the HTTP request
\item[\code{folderPath}] a string specifying the \code{folderPath} for the HTTP request
\item[\code{schemaName}] a string specifying the  \code{schemaName} for the HTTP request
\item[\code{sql}] a string containing the \code{sql} commands to be executed
\item[\code{maxRows}] (optional) an integer specifying how many rows of data to return. If no value is specified, all rows are returned.
\item[\code{rowOffset}] (optional) an integer specifying which row of data should be the first row in the retrieval. 
If no value is specified, all rows are returned.
\item[\code{stripAllHidden}] (optional) a logical value indicating whether or not to save data columns that would 
normally be hidden from user veiw. If no value is specified, no hidden columns are returned.
\end{ldescription}
\end{Arguments}
\begin{Details}\relax
A full dataset or user saved view can be imported into an R data frame using the \code{labkey.executeSql}
function. The function accepts as its arguments components of the url that identify the location of the
data and what Sql actions should be taken on the data prior to import. Data are returned in a data frame
with column names as they appear in on the labkey database website.
\end{Details}
\begin{Value}
The requested data are returned in a data frame.
\end{Value}
\begin{Author}\relax
Valerie Obenchain
\end{Author}
\begin{References}\relax
http://www.omegahat.org/RCurl/, 
http://dssm.unipa.it/CRAN/web/packages/rjson/rjson.pdf,
https://www.labkey.org/project/home/begin.view
\end{References}
\begin{SeeAlso}\relax
\code{\LinkA{labkey.selectRows}{labkey.selectRows}}
\end{SeeAlso}
\begin{Examples}
\begin{ExampleCode}

library(Rlabkey)

# Retrieve participant id, visit date and hemoglobin from Lab Results table
# from www.labkey.org
### NOTE: This won't work until 8.3 is up on www.labkey.org ####

#mydata <- labkey.executeSql(baseUrl="https://www.labkey.org", folderPath="/home/Study/demo", schemaName="study", 
#                               sql= 'select "Lab Results".ParticipantId, "Lab Results".Labdt, "Lab Results".Labhemo from "Lab Results"')

\end{ExampleCode}
\end{Examples}

\HeaderA{labkey.insertRows}{Insert new rows of data into a labkey database}{labkey.insertRows}
\keyword{IO}{labkey.insertRows}
\begin{Description}\relax
Send data from an R session to a labkey server to insert new rows of data in the database.
\end{Description}
\begin{Usage}
\begin{verbatim}
labkey.insertRows(baseUrl, folderPath, schemaName, queryName, toInsert, stripAllHidden = TRUE)
\end{verbatim}
\end{Usage}
\begin{Arguments}
\begin{ldescription}
\item[\code{baseUrl}] a string specifying the \code{baseUrl}for the HTTP request
\item[\code{folderPath}] a string specifying the \code{folderPath} for the HTTP request
\item[\code{schemaName}] a string specifying the  \code{schemaName} for the HTTP request
\item[\code{queryName}] a string specifying the  \code{queryName} for the HTTP request
\item[\code{toInsert}] a list containing field names and the corresponding data values to be inserted
\item[\code{stripAllHidden}] [optional] a logical value indicating whether or not to save data columns that would 
normally be hidden from user veiw. If no value is specified, no hidden columns are returned.
\end{ldescription}
\end{Arguments}
\begin{Details}\relax
A single row or multiple rows of data can be inserted at a time.
\end{Details}
\begin{Value}
Information returned to the user will include the \code{schemaName} and the \code{queryName} used in the insert
as well as the number of rows affected and the data sent.
\end{Value}
\begin{Author}\relax
Valerie Obenchain
\end{Author}
\begin{References}\relax
http://www.omegahat.org/RCurl/, 
http://dssm.unipa.it/CRAN/web/packages/rjson/rjson.pdf,
https://www.labkey.org/project/home/begin.view
\end{References}
\begin{SeeAlso}\relax
\code{\LinkA{labkey.selectRows}{labkey.selectRows}}
\end{SeeAlso}

\HeaderA{labkey.selectRows}{Retrieve data from a labkey database using url specifications}{labkey.selectRows}
\keyword{IO}{labkey.selectRows}
\begin{Description}\relax
Use url to specify data to be imported into R. Prior to import, data columns
can be sorted, specific columns or number of rows can be requested and
data filters can be applied.
\end{Description}
\begin{Usage}
\begin{verbatim}
labkey.selectRows(baseUrl, folderPath, schemaName, queryName, viewName = NULL, 
                                  colSelect = NULL, maxRows = NULL, rowOffset = NULL, colSort = NULL, 
                                  colFilter = NULL, stripAllHidden = TRUE)
\end{verbatim}
\end{Usage}
\begin{Arguments}
\begin{ldescription}
\item[\code{baseUrl}] a string specifying the \code{baseUrl}for the HTTP request
\item[\code{folderPath}] a string specifying the \code{folderPath} for the HTTP request
\item[\code{schemaName}] a string specifying the  \code{schemaName} for the HTTP request
\item[\code{queryName}] a string specifying the \code{queryName} for the HTTP request
\item[\code{viewName}] (optional) a string specifying the \code{viewName} for the HTTP request
\item[\code{colSelect}] (optional) a vector of comma separated strings specifying which columns of a dataset or view to import
\item[\code{maxRows}] (optional) an integer specifying how many rows of data to return. If no value is specified, all rows are returned.
\item[\code{colSort}] (optional) a string including the name of the column to sort preceeded by a "+" or "-" to indicate sort direction
\item[\code{rowOffset}] (optional) an integer specifying which row of data should be the first row in the retrieval. If no
value is specified, the retrieval starts with the first row.
\item[\code{colFilter}] (optional) a vector or array object created by the \code{makeFilter} function which contains the
column name, operator and value of the filter(s) to be applied to the retrieved data.
\item[\code{stripAllHidden}] (optional) a logical value indicating whether or not to save data columns that would normally be hiddenfrom user view. If no value is specified, no hidden columns are returned.
\end{ldescription}
\end{Arguments}
\begin{Details}\relax
A full dataset or user saved view can be imported into an R data frame using the \code{labkey.selectRows} 
function. The function accepts as its arguments the components of the url that identify
the location of the data and what actions should be taken on the data prior to import
(ie, sorting, selecting particular columns or maximum number of rows, etc.) Data are returned in a data 
frame with column names as they appear on the labkey database website. 

Use care when specifying column names for the colSelect or colFilter arguments. Often the column name
is not the same as the column header as seen on the web site. ***More help here*******

When importing data from ATLAS.scharp.org, a quick and simple way to identify the necessary components of the url 
(ie, schemaName, queryName, viewName, etc.) is to use the "export to R script" option avaiable as a drop down
under the "views" tab for each dataset.
\end{Details}
\begin{Value}
The requested data are returned in a data frame.
\end{Value}
\begin{Author}\relax
Valerie Obenchain
\end{Author}
\begin{References}\relax
http://www.omegahat.org/RCurl/, 
http://dssm.unipa.it/CRAN/web/packages/rjson/rjson.pdf,
https://www.labkey.org/project/home/begin.view
\end{References}
\begin{SeeAlso}\relax
\code{\LinkA{labkey.executeSql}{labkey.executeSql}}, \code{\LinkA{makeFilter}{makeFilter}}
\end{SeeAlso}
\begin{Examples}
\begin{ExampleCode}

## Retrieving data from the Labkey.org web site:

library(Rlabkey)

# Retrieve HIV Test Results and plot Western Blot data
getdata <- labkey.selectRows(baseUrl="https://www.labkey.org", folderPath="/home/Study/demo", 
                                schemaName="study", queryName="HIV Test Results")
plot(factor(getdata$"HIV Western Blot"), main="HIV Western Blot")

# Select columns and apply filters
myfilters<- makeFilter(c("HIVLoadQuant","GREATER_THAN",500), c("HIVRapidTest","EQUALS","Positive"))
getdata <- labkey.selectRows(baseUrl="https://www.labkey.org", folderPath="/home/Study/demo", schemaName="study", queryName="HIV Test Results", colSelect=c("ParticipantId","HIVDate","HIVLoadQuant","HIVRapidTest"), colFilter=myfilters)


\end{ExampleCode}
\end{Examples}

\HeaderA{labkey.updateRows}{Update rows of data in a labkey database}{labkey.updateRows}
\keyword{IO}{labkey.updateRows}
\begin{Description}\relax
Send data from an R session to a labkey server to update rows of data in the database.
\end{Description}
\begin{Usage}
\begin{verbatim}
labkey.updateRows(baseUrl, folderPath, schemaName, queryName, toUpdate, stripAllHidden = TRUE)
\end{verbatim}
\end{Usage}
\begin{Arguments}
\begin{ldescription}
\item[\code{baseUrl}] a string specifying the \code{baseUrl}for the HTTP request
\item[\code{folderPath}] a string specifying the \code{folderPath} for the HTTP request
\item[\code{schemaName}] a string specifying the  \code{schemaName} for the HTTP request
\item[\code{queryName}] a string specifying the  \code{queryName} for the HTTP request
\item[\code{toUpdate}] a list containing the name of the field and the corresponding data values to be updated
\item[\code{stripAllHidden}] [optional] a logical value indicating whether or not to save data columns that would 
normally be hidden from user veiw. If no value is specified, no hidden columns are returned.
\end{ldescription}
\end{Arguments}
\begin{Details}\relax
A single row or multiple rows of data can be updated at a time.
\end{Details}
\begin{Value}
Information returned to the user will include the \code{schemaName} and the \code{queryName} used in the update
as well as the number of rows affected and the data sent in the update.
\end{Value}
\begin{Author}\relax
Valerie Obenchain
\end{Author}
\begin{References}\relax
http://www.omegahat.org/RCurl/, 
http://dssm.unipa.it/CRAN/web/packages/rjson/rjson.pdf,
https://www.labkey.org/project/home/begin.view
\end{References}
\begin{SeeAlso}\relax
\code{\LinkA{labkey.selectRows}{labkey.selectRows}}
\end{SeeAlso}

\HeaderA{makeFilter}{Builds an array of filters}{makeFilter}
\keyword{file}{makeFilter}
\begin{Description}\relax
This function takes inputs of column name, filter value and filter operator for
the data to be filtered on. It returns an array of filters to be used in \code{labkey.selectRows}
\end{Description}
\begin{Usage}
\begin{verbatim}
makeFilter(c("colname", "operator",value))
\end{verbatim}
\end{Usage}
\begin{Arguments}
\begin{ldescription}
\item[\code{colname}] a string specifying the name of the column to be filtered
\item[\code{operator}] a text string specifying what operator should be used in the filter
\item[\code{value}] an integer or string specifying the value the columns should be filtered on
\end{ldescription}
\end{Arguments}
\begin{Details}\relax
Possible operator values are as follows:
"EQUALS", "NOT\_EQUALS", "GREATER\_THAN", "GREATER\_THAN\_OR\_EQUAL\_TO", "LESS\_THAN",
"LESS\_THAN\_OR\_EQUAL\_TO", "DATE\_EQUAL", "DATE\_NOT\_EQUAL", "NOT\_EQUAL\_OR\_NULL",
"IS\_NULL", "IS\_NOT\_NULL", "CONTAINS", and "DOES\_NOT\_CONTAIN".

Multiple filters can be applied (see examples). Currently this function supports
specifying up to five filters.
\end{Details}
\begin{Value}
The function returns either a single string or an array of strings to be use in the
\code{colFilter} argument of the \code{labkey.selectRows} function.
\end{Value}
\begin{Author}\relax
Valerie Obenchain
\end{Author}
\begin{References}\relax
http://www.omegahat.org/RCurl/, 
http://dssm.unipa.it/CRAN/web/packages/rjson/rjson.pdf,
https://www.labkey.org/project/home/begin.view
\end{References}
\begin{SeeAlso}\relax
\code{\LinkA{labkey.selectRows}{labkey.selectRows}}
\end{SeeAlso}
\begin{Examples}
\begin{ExampleCode}

# Specification of two filters:
myfilters<- makeFilter(c("HIVLoadQuant","GREATER_THAN",500), c("HIVRapidTest","EQUALS","Positive"))

# Filter using "equals one of" operator:
myfilter2 <- makeFilter(filter1=c("HIVLoadIneq","EQUALS_ONE_OF","Equals ; Less than"))

# Use in labkey.selectRows function
getdata <- labkey.selectRows(baseUrl="https://www.labkey.org", folderPath="/home/Study/demo", schemaName="study", queryName="HIV Test Results", colSelect=c("ParticipantId","HIVDate","HIVLoadQuant","HIVRapidTest"), colFilter=myfilters)


\end{ExampleCode}
\end{Examples}

\end{document}
