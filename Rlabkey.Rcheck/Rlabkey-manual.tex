\documentclass{article}
\usepackage[ae,hyper]{Rd}
\begin{document}
\HeaderA{Rlabkey-package}{Import/export data between a labkey database and R}{Rlabkey.Rdash.package}
\aliasA{Rlabkey}{Rlabkey-package}{Rlabkey}
\keyword{package}{Rlabkey-package}
\begin{Description}\relax
This package allows the transfer of data between a labkey database and an R session. Data can be imported from 
a labkey database into R by specifying the query schema information (\code{labkey.selectRows}) 
or by using sql commands (\code{labkey.executeSql}). From an R session, existing data can be updated
(\code{labkey.updateRows}), new data can be inserted (\code{labkey.insertRows}) or  
data can be deleted from the labkey database (\code{labkey.deleteRows}). 

The user must have the appropriate authorization on the labkey
server in order to modify the database through the use of
these functions.
\end{Description}
\begin{Details}\relax
\Tabular{ll}{
Package: & Rlabkey\\
Type: & Package\\
Version: & 0.0.3\\
Date: & 2008-09-02\\
License: & Apache 2.0\\
LazyLoad: & yes\\
}
Using this package to access a password protected labkey data base requires that the user
has their login information in a netrc file. The netrc file
contains configuration and autologin information for the File Transfer Protocol client (ftp) and
other programs such as CURL.

On a UNIX system this file should be named .netrc (dot netrc) and on windows it sould be 
named \_netrc (underscore netrc). The file should be located in the users home directory and the 
permissions on the file should be unreadable for everybody except the owner.  

To create the \_netrc on a windows machine, first create an environment variable called 'HOME' that 
is set to your home directory (c:/Users/<User-Name> on Vista) or any directory you want to use. 
In that directory, create a text file named \_netrc (note that it's underscore netrc, not dot 
netrc like it is on UNIX). 

The following three lines must be included in the .netrc or \_netrc file either separated by white space
(spaces, tabs, or newlines) or commas.

machine <remote-machine-name>\\
login <user-email>\\
password <user-password>


One example would be:\\
machine atlas.scharp.org\\
login vobencha@fhcrc.org\\
password mypassword\\

Another example would be:\\
machine atlas.scharp.org login vobencha@fhcrc.org password mypassword
\end{Details}
\begin{Author}\relax
Valerie Obenchain
\end{Author}
\begin{References}\relax
http://www.omegahat.org/RCurl/,\\
http://dssm.unipa.it/CRAN/web/packages/rjson/rjson.pdf,\\
https://www.labkey.org/project/home/begin.view
\end{References}
\begin{SeeAlso}\relax
\code{\LinkA{labkey.selectRows}{labkey.selectRows}}, \code{\LinkA{labkey.executeSql}{labkey.executeSql}}, \code{\LinkA{makeFilter}{makeFilter}}, 
\code{\LinkA{labkey.insertRows}{labkey.insertRows}}, \code{\LinkA{labkey.updateRows}{labkey.updateRows}}, \code{\LinkA{labkey.deleteRows}{labkey.deleteRows}}
\end{SeeAlso}

\HeaderA{labkey.deleteRows}{Delete rows of data from a labkey database}{labkey.deleteRows}
\keyword{IO}{labkey.deleteRows}
\begin{Description}\relax
Specify rows of data to be deleted from the database.
\end{Description}
\begin{Usage}
\begin{verbatim}
labkey.deleteRows(baseUrl, folderPath, schemaName, queryName, toDelete, 
stripAllHidden = TRUE)
\end{verbatim}
\end{Usage}
\begin{Arguments}
\begin{ldescription}
\item[\code{baseUrl}] a string specifying the \code{baseUrl}for labkey server
\item[\code{folderPath}] a string specifying the \code{folderPath}
\item[\code{schemaName}] a string specifying the  \code{schemaName} for the query
\item[\code{queryName}] a string specifying the  \code{queryName}
\item[\code{toDelete}] a data frame containing a single column of data containing the data identifiers of the rows to be deleted
\item[\code{stripAllHidden}] (optional) a logical value indicating whether or not to return data columns that would 
normally be hidden from user veiw. If no value is specified, no hidden columns are returned.
\end{ldescription}
\end{Arguments}
\begin{Details}\relax
A single row or multiple rows of data can be deleted.  The \code{toDelete} data frame should consist of a single 
column of data containing the data identifiers of the rows to be deleted (e.g., lsid).
The data frame must be created with the \code{stringsAsFactors} set to FALSE.

NOTE: Each variable in a dataset has both a column label and a column name. The column label is visable at the top
of each column on the web page and is longer and more descriptive. The column name is shorter and is 
used \dQuote{behind the scenes} for database manipulation. It is the column name that must be used in
the Rlabkey functions when a column name is expected. To identify a particular column name in a dataset on
a web site, use the \dQuote{export to R script} option available as a drop down option under the \dQuote{views} 
tab for each dataset.
\end{Details}
\begin{Value}
A list is returned with named categories of \bold{command}, \bold{rowsAffected}, \bold{rows}, \bold{queryName}, \bold{containerPath} and \bold{schemaName}.
The \bold{schemaName}, \bold{queryName} and \bold{containerPath} properties contain the same schema, query 
and folder path used in the request.  The
\bold{rowsAffected} property indicates he number of rows affected by the API action. This will typically be the same
number as passed in the request. The \bold{rows} property contains a list of rows corresponding to the rows
deleted.
\end{Value}
\begin{Author}\relax
Valerie Obenchain
\end{Author}
\begin{References}\relax
http://www.omegahat.org/RCurl/, \\
http://dssm.unipa.it/CRAN/web/packages/rjson/rjson.pdf,\\
https://www.labkey.org/project/home/begin.view
\end{References}
\begin{SeeAlso}\relax
\code{\LinkA{labkey.selectRows}{labkey.selectRows}}, \code{\LinkA{labkey.executeSql}{labkey.executeSql}}, \code{\LinkA{makeFilter}{makeFilter}}, 
\code{\LinkA{labkey.insertRows}{labkey.insertRows}}, \code{\LinkA{labkey.updateRows}{labkey.updateRows}}
\end{SeeAlso}
\begin{Examples}
\begin{ExampleCode}

## Delete two rows of data
#delrows <- data.frame(lsid=c('urn:lsid:labkey.org:****','urn:lsid:labkey.org:****',
#'urn:lsid:labkey.org.****'),stringsAsFactors=FALSE))

#labkey.deleteRows(     baseUrl="https://www.labkey.org", 
#                                       folderPath="/home/Study/demo", 
#                                       schemaName="study",     
#                                       queryName="HIV Test Results",   
#                                       toDelete=delrows)

\end{ExampleCode}
\end{Examples}

\HeaderA{labkey.executeSql}{Retrieve data from a labkey database using SQL commands}{labkey.executeSql}
\keyword{IO}{labkey.executeSql}
\begin{Description}\relax
Use Sql commands to specify data to be imported into R. Prior to import, data can
be manipulated through standard SQL commands supported in labkey SQL.
\end{Description}
\begin{Usage}
\begin{verbatim}
labkey.executeSql(baseUrl, folderPath, schemaName, sql, maxRows = NULL, 
rowOffset = NULL, stripAllHidden = TRUE)
\end{verbatim}
\end{Usage}
\begin{Arguments}
\begin{ldescription}
\item[\code{baseUrl}] a string specifying the \code{baseUrl}for the labkey server
\item[\code{folderPath}] a string specifying the \code{folderPath} 
\item[\code{schemaName}] a string specifying the  \code{schemaName} for the query
\item[\code{sql}] a string containing the \code{sql} commands to be executed
\item[\code{maxRows}] (optional) an integer specifying the maximum number of rows to return. If no value is specified, all rows are returned.
\item[\code{rowOffset}] (optional) an integer specifying which row of data should be the first row in the retrieval. 
If no value is specified, rows will begin at the start of the result set.
\item[\code{stripAllHidden}] (optional) a logical value indicating whether or not to return data columns that would 
normally be hidden from user veiw. If no value is specified, no hidden columns are returned.
\end{ldescription}
\end{Arguments}
\begin{Details}\relax
Rows returned from the SQL query are imported into an R data frame using the \code{labkey.executeSql}
function. Function arguments are components of the url that identify the location of the
data and what SQL actions should be taken on the data prior to import.


NOTE: Each variable in a dataset has both a column label and a column name. The column label is visable at the top
of each column on the web page and is longer and more descriptive. The column name is shorter and is
used \dQuote{behind the scenes} for database manipulation. It is the column name that must be used in
the Rlabkey functions when a column name is expected. To identify a particular column name in a dataset on
a web site, use the \dQuote{export to R script} option available as a drop down option under the \dQuote{views}
tab for each dataset.
\end{Details}
\begin{Value}
The requested data are returned in a data frame with column names as they appear on the website.
\end{Value}
\begin{Author}\relax
Valerie Obenchain
\end{Author}
\begin{References}\relax
http://www.omegahat.org/RCurl/,\\ 
http://dssm.unipa.it/CRAN/web/packages/rjson/rjson.pdf,\\
https://www.labkey.org/project/home/begin.view
\end{References}
\begin{SeeAlso}\relax
\code{\LinkA{labkey.selectRows}{labkey.selectRows}}, \code{\LinkA{makeFilter}{makeFilter}}, \code{\LinkA{labkey.insertRows}{labkey.insertRows}}, 
\code{\LinkA{labkey.updateRows}{labkey.updateRows}}, \code{\LinkA{labkey.deleteRows}{labkey.deleteRows}}
\end{SeeAlso}
\begin{Examples}
\begin{ExampleCode}

### Select participants who meet acute status requirements
#getacute <- labkey.executeSql(baseUrl="https://www.labkey.org",
#                            folderPath="/home/Study/demo",
#                            schemaName="study",
#                            sql = 'select "Status Assessment".ParticipantId from "Status Assessment" where "Status Asses#sment"."StatusMeetCriteria"=\'yes\'')
#
#
### Average ages over different gender groups
#getage <- labkey.executeSql(baseUrl="https://www.labkey.org",
#                            folderPath="/home/Study/demo",
#                            schemaName="study",
#                            sql = "select Demographics.Gender, avg(Demographics.Age) as Number from Demographics group b#y Demographics.Gender")
#

### Select data for participants with partner information 
#getpartners <- labkey.executeSql(baseUrl="https://www.labkey.org",
#                            folderPath="/home/Study/demo",
#                            schemaName="study",
#                            sql = 'select "Status Assessment".StatusPartner1 from "Status Assessment" where "Status Asse#ssment".StatusPartner1 is not null')
#



\end{ExampleCode}
\end{Examples}

\HeaderA{labkey.insertRows}{Insert new rows of data into a labkey database}{labkey.insertRows}
\keyword{IO}{labkey.insertRows}
\begin{Description}\relax
Insert new rows of data into the database.
\end{Description}
\begin{Usage}
\begin{verbatim}
labkey.insertRows(baseUrl, folderPath, schemaName, queryName, toInsert, 
stripAllHidden = TRUE)
\end{verbatim}
\end{Usage}
\begin{Arguments}
\begin{ldescription}
\item[\code{baseUrl}] a string specifying the \code{baseUrl}for the labkey server
\item[\code{folderPath}] a string specifying the \code{folderPath} 
\item[\code{schemaName}] a string specifying the  \code{schemaName} for the query
\item[\code{queryName}] a string specifying the  \code{queryName} 
\item[\code{toInsert}] a data frame containing rows of data to be inserted
\item[\code{stripAllHidden}] (optional) a logical value indicating whether or not to return data columns that would 
normally be hidden from user veiw. If no value is specified, no hidden columns are returned.
\end{ldescription}
\end{Arguments}
\begin{Details}\relax
A single row or multiple rows of data can be inserted.  The \code{toInsert} data frame must contain
values for each column in the dataset and must be created with the \code{stringsAsFactors} option
set to FALSE. When inserting data into a study dataset, the sequence number must be specified.

NOTE: Each variable in a dataset has both a column label and a column name. The column label is visable at the top
of each column on the web page and is longer and more descriptive. The column name is shorter and is
used \dQuote{behind the scenes} for database manipulation. It is the column name that must be used in
the Rlabkey functions when a column name is expected. To identify a particular column name in a dataset on
a web site, use the \dQuote{export to R script} option available as a drop down option under the \dQuote{views}
tab for each dataset.
\end{Details}
\begin{Value}
A list is returned with named categories of \bold{command}, \bold{rowsAffected}, \bold{rows}, \bold{queryName}, \bold{containerPath} and \bold{schemaName}.
The \bold{schemaName}, \bold{queryName} and \bold{containerPath} properties contain the same schema, query 
and folder path used in the request.  The
\bold{rowsAffected} property indicates he number of rows affected by the API action. This will typically be the same
number as passed in the request. The \bold{rows} property contains a list of row objects corresponding to the rows 
inserted.
\end{Value}
\begin{Author}\relax
Valerie Obenchain
\end{Author}
\begin{References}\relax
http://www.omegahat.org/RCurl/,\\ 
http://dssm.unipa.it/CRAN/web/packages/rjson/rjson.pdf,\\
https://www.labkey.org/project/home/begin.view
\end{References}
\begin{SeeAlso}\relax
\code{\LinkA{labkey.selectRows}{labkey.selectRows}}, \code{\LinkA{labkey.executeSql}{labkey.executeSql}}, \code{\LinkA{makeFilter}{makeFilter}}, 
\code{\LinkA{labkey.updateRows}{labkey.updateRows}}, \code{\LinkA{labkey.deleteRows}{labkey.deleteRows}}
\end{SeeAlso}
\begin{Examples}
\begin{ExampleCode}

## Insert two rows of data:
#newrows <- data.frame(participantID=c(24932540, 24932541), SequenceNum=c(2,3), age=c(40, 25), 
#height=c(70,65), gender=c("m","f"), city=c("Boston","New York"), state=c("MA","NY"), 
#country=c("USA","USA"), stringsAsFactors=FALSE)

#labkey.insertRows(     baseUrl="https://www.labkey.org",       
#                                       folderPath="/home/Study/demo", 
#                                       schemaName="study",     
#                                       queryName="Demographics", 
#                                       toInsert=newrows)


\end{ExampleCode}
\end{Examples}

\HeaderA{labkey.selectRows}{Retrieve data from a labkey database}{labkey.selectRows}
\keyword{IO}{labkey.selectRows}
\begin{Description}\relax
Import full datasets or selected rows into R. The data can be sorted and filtered prior to import.
\end{Description}
\begin{Usage}
\begin{verbatim}
labkey.selectRows(baseUrl, folderPath, schemaName, queryName, viewName = NULL, 
colSelect = NULL, maxRows = NULL, rowOffset = NULL, colSort = NULL, 
colFilter = NULL, stripAllHidden = TRUE)
\end{verbatim}
\end{Usage}
\begin{Arguments}
\begin{ldescription}
\item[\code{baseUrl}] a string specifying the \code{baseUrl}for the labkey server
\item[\code{folderPath}] a string specifying the \code{folderPath} 
\item[\code{schemaName}] a string specifying the  \code{schemaName} for the query
\item[\code{queryName}] a string specifying the \code{queryName}
\item[\code{viewName}] (optional) a string specifying the \code{viewName}
\item[\code{colSelect}] (optional) a vector of comma separated strings specifying which columns of a dataset or view to import
\item[\code{maxRows}] (optional) an integer specifying how many rows of data to return. If no value is specified, all rows are returned.
\item[\code{colSort}] (optional) a string including the name of the column to sort preceeded by a \dQuote{+} or
\dQuote{-} to indicate sort direction
\item[\code{rowOffset}] (optional) an integer specifying which row of data should be the first row in the retrieval. If no
value is specified, the retrieval starts with the first row.
\item[\code{colFilter}] (optional) a vector or array object created by the \code{makeFilter} function which contains the
column name, operator and value of the filter(s) to be applied to the retrieved data.
\item[\code{stripAllHidden}] (optional) a logical value indicating whether or not to return data columns that would normally be hiddenfrom user view. If no value is specified, no hidden columns are returned.
\end{ldescription}
\end{Arguments}
\begin{Details}\relax
A full dataset or user saved view can be imported into an R data frame using the 
\code{labkey.selectRows} function. Function arguments are the components of the url that identify
the location of the data and what actions should be taken on the data prior to import
(ie, sorting, selecting particular columns or maximum number of rows, etc.).

NOTE: Each variable in a dataset has both a column label and a column name. The column label is visable at the top
of each column on the web page and is longer and more descriptive. The column name is shorter and is
used \dQuote{behind the scenes} for database manipulation. It is the column name that must be used in
the Rlabkey functions when a column name is expected. To identify a particular column name in a dataset on
a web site, use the \dQuote{export to R script} option available as a drop down option under the \dQuote{views}
tab for each dataset.
\end{Details}
\begin{Value}
The requested data are returned in a data frame with column names as they appear on the website.
\end{Value}
\begin{Author}\relax
Valerie Obenchain
\end{Author}
\begin{References}\relax
http://www.omegahat.org/RCurl/,\\ 
http://dssm.unipa.it/CRAN/web/packages/rjson/rjson.pdf,\\
https://www.labkey.org/project/home/begin.view
\end{References}
\begin{SeeAlso}\relax
\code{\LinkA{labkey.executeSql}{labkey.executeSql}}, \code{\LinkA{makeFilter}{makeFilter}}, \code{\LinkA{labkey.insertRows}{labkey.insertRows}}, 
\code{\LinkA{labkey.updateRows}{labkey.updateRows}}, \code{\LinkA{labkey.deleteRows}{labkey.deleteRows}}
\end{SeeAlso}
\begin{Examples}
\begin{ExampleCode}

# Retrieve HIV Test Results and plot Western Blot data
#getdata <- labkey.selectRows(  baseUrl="https://www.labkey.org", 
#                                                               folderPath="/home/Study/demo", 
#                                                               schemaName="study", 
#                                                               queryName="HIV Test Results")
#plot(factor(getdata$"HIV Western Blot"), main="HIV Western Blot")

# Select columns and apply filters
#myfilters<- makeFilter(c("HIVLoadQuant","GREATER_THAN",500), c("HIVRapidTest","EQUALS","Positive"))

#getdata <- labkey.selectRows(  baseUrl="https://www.labkey.org", 
#                                                               folderPath="/home/Study/demo", 
#                                                               schemaName="study",     
#                                                               queryName="HIV Test Results", 
#                                                               colSelect=c("ParticipantId","HIVDate","HIVLoadQuant","HIVRapidTest"), 
#                                                               colFilter=myfilters)


\end{ExampleCode}
\end{Examples}

\HeaderA{labkey.updateRows}{Update existing rows of data in a labkey database}{labkey.updateRows}
\keyword{IO}{labkey.updateRows}
\begin{Description}\relax
Send data from an R session to update existing rows of data in the database.
\end{Description}
\begin{Usage}
\begin{verbatim}
labkey.updateRows(baseUrl, folderPath, schemaName, queryName, toUpdate, 
stripAllHidden = TRUE)
\end{verbatim}
\end{Usage}
\begin{Arguments}
\begin{ldescription}
\item[\code{baseUrl}] a string specifying the \code{baseUrl}for the labkey server
\item[\code{folderPath}] a string specifying the \code{folderPath} 
\item[\code{schemaName}] a string specifying the  \code{schemaName}for the query
\item[\code{queryName}] a string specifying the  \code{queryName}
\item[\code{toUpdate}] a data frame containing the row(s) of data to be updated
\item[\code{stripAllHidden}] (optional) a logical value indicating whether or not to return data columns that would 
normally be hidden from user veiw. If no value is specified, no hidden columns are returned.
\end{ldescription}
\end{Arguments}
\begin{Details}\relax
A single row or multiple rows of data can be updated.  The \code{toUpdate} data frame should contain 
the rows of data to be updated and must be created with the \code{stringsAsFactors} option
set to FALSE. 

NOTE: Each variable in a dataset has both a column label and a column name. The column label is visable at the top
of each column on the web page and is longer and more descriptive. The column name is shorter and is
used \dQuote{behind the scenes} for database manipulation. It is the column name that must be used in
the Rlabkey functions when a column name is expected. To identify a particular column name in a dataset on
a web site, use the \dQuote{export to R script} option available as a drop down option under the \dQuote{views}
tab for each dataset.
\end{Details}
\begin{Value}
A list is returned with named categories of \bold{command}, \bold{rowsAffected}, \bold{rows}, \bold{queryName}, \bold{containerPath} and \bold{schemaName}.
The \bold{schemaName}, \bold{queryName} and \bold{containerPath} properties contain the same schema, query 
and folder path used in the request.  The
\bold{rowsAffected} property indicates he number of rows affected by the API action. This will typically be the same
number as passed in the request. The \bold{rows} property contains a list of row objects corresponding to the rows
updated.
\end{Value}
\begin{Author}\relax
Valerie Obenchain
\end{Author}
\begin{References}\relax
http://www.omegahat.org/RCurl/, \\
http://dssm.unipa.it/CRAN/web/packages/rjson/rjson.pdf,\\
https://www.labkey.org/project/home/begin.view
\end{References}
\begin{SeeAlso}\relax
\code{\LinkA{labkey.selectRows}{labkey.selectRows}}, \code{\LinkA{labkey.executeSql}{labkey.executeSql}}, \code{\LinkA{makeFilter}{makeFilter}}, 
\code{\LinkA{labkey.insertRows}{labkey.insertRows}}, \code{\LinkA{labkey.deleteRows}{labkey.deleteRows}}
\end{SeeAlso}
\begin{Examples}
\begin{ExampleCode}

### Retrieve data from the database
#getdata <- labkey.selectRows(  baseUrl="https://www.labkey.org", 
#                                                               folderPath="/home/Study/demo", 
#                                                               schemaName="study", 
#                                                               queryName="Demographics")
#
### Modify the data
#modifyDat <- getdata[2,]
#modifyDat$City <- "Tacoma"
#
### Will need lsid for rows
### Update the rows in the database with the modified data
#labkey.updateRows(     baseUrl="https://www.labkey.org", 
#                                       folderPath="/home/Study/demo", 
#                                       schemaName="study", 
#                                       queryName="HIV Test Results", 
#                                       toUpdate=modifyDat)
#
#
\end{ExampleCode}
\end{Examples}

\HeaderA{makeFilter}{Builds an array of filters}{makeFilter}
\keyword{file}{makeFilter}
\begin{Description}\relax
This function takes inputs of column name, filter value and filter operator and 
returns an array of filters to be used in \code{labkey.selectRows}.
\end{Description}
\begin{Usage}
\begin{verbatim}
makeFilter(c(colname, operator,value))
\end{verbatim}
\end{Usage}
\begin{Arguments}
\begin{ldescription}
\item[\code{colname}] a string specifying the name of the column to be filtered
\item[\code{operator}] a string specifying what operator should be used in the filter (see options below)
\item[\code{value}] an integer or string specifying the value the columns should be filtered on
\end{ldescription}
\end{Arguments}
\begin{Details}\relax
These filters are applied to the data prior to import into R. Currently this function
allows the user to specify up to five filters. Multiple filters can be entered named or
un-named vectors (see examples below).

Possible operator values are as follows:
"EQUALS", "EQUALS\_ONE\_OF", "NOT\_EQUALS", "GREATER\_THAN", "GREATER\_THAN\_OR\_EQUAL\_TO", "LESS\_THAN",
"LESS\_THAN\_OR\_EQUAL\_TO", "DATE\_EQUAL", "DATE\_NOT\_EQUAL", "NOT\_EQUAL\_OR\_NULL",
"IS\_MISSING", "IS\_NOT\_MISSING", "CONTAINS", "DOES\_NOT\_CONTAIN", "STARTS\_WITH", and "DOES\_NOT\_START\_WITH".
\end{Details}
\begin{Value}
The function returns either a single string or an array of strings to be use in the
\code{colFilter} argument of the \code{labkey.selectRows} function.
\end{Value}
\begin{Author}\relax
Valerie Obenchain
\end{Author}
\begin{References}\relax
http://www.omegahat.org/RCurl/, \\
http://dssm.unipa.it/CRAN/web/packages/rjson/rjson.pdf,\\
https://www.labkey.org/project/home/begin.view
\end{References}
\begin{SeeAlso}\relax
\code{\LinkA{labkey.selectRows}{labkey.selectRows}}
\end{SeeAlso}
\begin{Examples}
\begin{ExampleCode}

# Specify two filters:
myfilters<- makeFilter(c("HIVLoadQuant","GREATER_THAN",500), c("HIVRapidTest","EQUALS","Positive"))

# Filter using "equals one of" operator:
myfilter2 <- makeFilter(filter1=c("HIVLoadIneq","EQUALS_ONE_OF","Equals ; Less than"))

# Use in labkey.selectRows function
getdata <- labkey.selectRows(baseUrl="https://www.labkey.org", folderPath="/home/Study/demo", 
schemaName="study", queryName="HIV Test Results", 
colSelect=c("ParticipantId","HIVDate","HIVLoadQuant","HIVRapidTest"), colFilter=myfilters)


\end{ExampleCode}
\end{Examples}

\end{document}
