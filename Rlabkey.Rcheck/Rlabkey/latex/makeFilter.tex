\HeaderA{makeFilter}{Builds an array of filters}{makeFilter}
\keyword{file}{makeFilter}
\begin{Description}\relax
This function takes inputs of column name, filter value and filter operator for
the data to be filtered on. It returns an array of filters to be used in \code{labkey.selectRows}
\end{Description}
\begin{Usage}
\begin{verbatim}
makeFilter(c("colname", "operator",value))
\end{verbatim}
\end{Usage}
\begin{Arguments}
\begin{ldescription}
\item[\code{colname}] a string specifying the name of the column to be filtered
\item[\code{operator}] a text string specifying what operator should be used in the filter
\item[\code{value}] an integer or string specifying the value the columns should be filtered on
\end{ldescription}
\end{Arguments}
\begin{Details}\relax
Possible operator values are as follows:
"EQUALS", "NOT\_EQUALS", "GREATER\_THAN", "GREATER\_THAN\_OR\_EQUAL\_TO", "LESS\_THAN",
"LESS\_THAN\_OR\_EQUAL\_TO", "DATE\_EQUAL", "DATE\_NOT\_EQUAL", "NOT\_EQUAL\_OR\_NULL",
"IS\_NULL", "IS\_NOT\_NULL", "CONTAINS", and "DOES\_NOT\_CONTAIN".

Multiple filters can be applied (see examples). Currently this function supports
specifying up to five filters.
\end{Details}
\begin{Value}
The function returns either a single string or an array of strings to be use in the
\code{colFilter} argument of the \code{labkey.selectRows} function.
\end{Value}
\begin{Author}\relax
Valerie Obenchain
\end{Author}
\begin{References}\relax
http://www.omegahat.org/RCurl/, 
http://dssm.unipa.it/CRAN/web/packages/rjson/rjson.pdf,
https://www.labkey.org/project/home/begin.view
\end{References}
\begin{SeeAlso}\relax
\code{\LinkA{labkey.selectRows}{labkey.selectRows}}
\end{SeeAlso}
\begin{Examples}
\begin{ExampleCode}

# Create filters
myfilters<- makeFilter(c("HIVLoadQuant","GREATER_THAN",500), c("HIVRapidTest","EQUALS","Positive"))

# Use in labkey.selectRows function
getdata <- labkey.selectRows(baseUrl="https://www.labkey.org", folderPath="/home/Study/demo", schemaName="study", queryName="HIV Test Results", colSelect=c("ParticipantId","HIVDate","HIVLoadQuant","HIVRapidTest"), colFilter=myfilters)


\end{ExampleCode}
\end{Examples}

